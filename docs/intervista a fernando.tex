% Options for packages loaded elsewhere
\PassOptionsToPackage{unicode}{hyperref}
\PassOptionsToPackage{hyphens}{url}
%
\documentclass[
]{book}
\usepackage{amsmath,amssymb}
\usepackage{iftex}
\ifPDFTeX
  \usepackage[T1]{fontenc}
  \usepackage[utf8]{inputenc}
  \usepackage{textcomp} % provide euro and other symbols
\else % if luatex or xetex
  \usepackage{unicode-math} % this also loads fontspec
  \defaultfontfeatures{Scale=MatchLowercase}
  \defaultfontfeatures[\rmfamily]{Ligatures=TeX,Scale=1}
\fi
\usepackage{lmodern}
\ifPDFTeX\else
  % xetex/luatex font selection
\fi
% Use upquote if available, for straight quotes in verbatim environments
\IfFileExists{upquote.sty}{\usepackage{upquote}}{}
\IfFileExists{microtype.sty}{% use microtype if available
  \usepackage[]{microtype}
  \UseMicrotypeSet[protrusion]{basicmath} % disable protrusion for tt fonts
}{}
\makeatletter
\@ifundefined{KOMAClassName}{% if non-KOMA class
  \IfFileExists{parskip.sty}{%
    \usepackage{parskip}
  }{% else
    \setlength{\parindent}{0pt}
    \setlength{\parskip}{6pt plus 2pt minus 1pt}}
}{% if KOMA class
  \KOMAoptions{parskip=half}}
\makeatother
\usepackage{xcolor}
\setlength{\emergencystretch}{3em} % prevent overfull lines
\providecommand{\tightlist}{%
  \setlength{\itemsep}{0pt}\setlength{\parskip}{0pt}}
\setcounter{secnumdepth}{-\maxdimen} % remove section numbering
\usepackage{bookmark}
\IfFileExists{xurl.sty}{\usepackage{xurl}}{} % add URL line breaks if available
\urlstyle{same}
\hypersetup{
  hidelinks,
  pdfcreator={LaTeX via pandoc}}

\author{}
\date{}

\begin{document}
\frontmatter

\mainmatter
\section{🎯 INTERVISTA STRATEGICA: LA TUA STORIA DI LEADERSHIP AUTENTICA}\label{intervista-strategica-la-tua-storia-di-leadership-autentica}

\emph{Compila questo modulo per costruire lo storytelling autobiografico che attraverserà tutto il libro}

\begin{center}\rule{0.5\linewidth}{0.5pt}\end{center}

\subsection{🌟 SEZIONE I: I MOMENTI FONDANTI}\label{sezione-i-i-momenti-fondanti}

\subsubsection{\texorpdfstring{\textbf{IL RISVEGLIO DELLA CONSAPEVOLEZZA}}{IL RISVEGLIO DELLA CONSAPEVOLEZZA}}\label{il-risveglio-della-consapevolezza}

\textbf{1. Il momento della verità} \emph{Descrivi il momento specifico in cui hai realizzato che la leadership non era quello che pensavi. Dov'eri? Cosa stava succedendo? Quale sensazione fisica hai provato?}

\textbf{RISPOSTA:}

\begin{verbatim}
ho letto molti libri, sono sempre stato interessato alla leadership perche da quando avevo 16 ho capito che il mio scopo nella vita era quello di creare un gruppo di persone che collaborando insieme potessero creare la differenza e cambiare una società che a 16 nel 1973 mi sembrava senza scopi, senza metodi di collaborare. Le persone mi sembravano ognuna per conto loro. Io pensavo che a quei tempi tutto potesse essere migliorato tramite la collaborazione. Poi, nella realtà dei fatti, furono gli anni di piombo a dettare l'agenda della società. Brigate rosse, prima linea, i gruppi anarchici, le flange neofasciste, la guerra di classe. Ed io mi ritirai a poco a poco dal mio desiderato protagonismo sociale. Ricordo che pensavo ma perchè non è possibile creare un gruppo di acquisto per abbattere i costi della vita, perchè le persone che mi circondano non riescono a comprendere quanto potrebbe essere facile cambiare le cose con un semplice approccio pragmatico che partisse dalla realtà per trasformarla piuttosto che cercare di abbatterla. Avevo 16 anni e mio malgrado questo mio sogno rimase tale per molti , molti anni.
\end{verbatim}

\textbf{2. Il primo fallimento significativo} \emph{Qual è stato il tuo primo ``fallimento di leadership'' significativo? Cosa hai fatto di sbagliato e, soprattutto, cosa ti sei detto in quel momento?}

\textbf{RISPOSTA:}

\begin{verbatim}
Mi ricordo che quello fu il mio vero fallimento; allora ho scoperto che non basta avere delle idee per diventare leader ovvero coordinare le menti e i corpi delle persone per un maggior bene comune. Dentro di me sentivo una morsa che mi stringeva il cuore; non potevo pensare al fatto che le cose che io consideravo importanti per me stesso e per gli altri non fosse possibile veicolarlo verso gli altri, raggruppandone gli intenti verso uno scopo pratico comune e raggiungibie. Notavo che tutti quelli che insontravo si trovassero in una spece di loro sogno e forse anche il mio era solo un sogno irrealizzabile. Fatto sta che da quella grossa deluzione non tiuscii veramente a risalire la china fino al 1982. Furono 9 anni di alienazione sociale, ricerca edonica del piacere, della perdita del tempo. Saltavo da una relazione ad un altra, da un amicizia ad un altra. Tanti e tanti gruppi di persone diverse appartenenti a ceti sociali diversi che tra loro avevano poco in comune se non il desideirio di perdersi nella loro impossibilità ad essere incisivi nella loro vita. RIcordo un episodio in cui io ed alcuni amici stessi mo suonando e cantando su una collinetta e incortrammo una persona della società bene di torino. Ci sembrò interessante l'incontro ma per me quell'episodio fu una specie di salto di paradigma. entrai in contatto con figli di industriali ed immobiliaristi, la crema della socità. pensai adesso posso imparare qualcosa... ma la delusione fu incontrare un degrado spirituale ancora maggiore. Atteggiamenti infimi, depravazioni, bullismo, ma con i soldi. Ma questo gruppo sociale ri rivelo ancora peggiore degli altri
\end{verbatim}

\textbf{3. Aspettando il permesso} \emph{C'è stato un momento in cui hai pensato ``Sto aspettando che qualcuno mi dia il permesso di essere leader''? Racconta quella situazione.}

\textbf{RISPOSTA:}

\begin{verbatim}
ricordo che in una cabina telefonica stavo parlando con un manager, penso fosse intorno al 1985 o 1986 in una frresca giornata dis sole primaverile. Chiedevo quasi il permesso di essere la persona lungimirante che ero, il pragmatico attivo capace di cogliere ed usare le sottigliezze del mercato. RIcordo che perso la pazienza non non sentendomi apprezzato per quello che sapec+vo e sentivo di me. Parlavo di soluzioni ricevevo impossibilità, parlavo di prospettive ricevevo continue insinuazioni rispetto alla mia inconsistenza basata sui fatti. In effetti mi ero relegato ad un ruolo di comprimario, ti persona volenterosa a cui far fare le cose che altri non vogliono fare. Un quel momento tutto il peso delle impossibilità venivano rappresentate dallo spazio angusto della cabina telefonica. Volevo far arrivare il mio messaggio e l'unico modo che pensavo utile era alzare la vodce, diventare arrogante ed insensibile tutte caratterisctiche che sapevo non essere ne utili ne necessarie. era come sentirsi in un precipizio con le mani nelle mani di un altro e chiedergli tirami su, portami al riparo da questa situazione pericolosa in cui mi trovo. sentivo la roccia sgretolarsi sotto i piedi ogni volta che provavo ad appoggiare il piede su un appiglio. quello fu veramente un momento "indimenticabile" della sofferenza che oramai avevo quasi sclerotizzata nell'animo e a cui mi riglellavo con antagonismo e disperazione.
\end{verbatim}

\begin{center}\rule{0.5\linewidth}{0.5pt}\end{center}

\subsection{⚡ SEZIONE II: GLI ERRORI CHE TI HANNO FORMATO}\label{sezione-ii-gli-errori-che-ti-hanno-formato}

\subsubsection{\texorpdfstring{\textbf{LE TUE LEZIONI PIÙ COSTOSE}}{LE TUE LEZIONI PIÙ COSTOSE}}\label{le-tue-lezioni-piuxf9-costose}

\textbf{4. L'errore più costoso} \emph{Racconta l'errore di leadership più costoso che hai mai fatto (in termini economici, relazionali o di opportunità perse). Quanto ti è ``costato'' e cosa hai imparato?}

\textbf{RISPOSTA:}

\begin{verbatim}
[ricordo che lavoravo nel settore delle energie rinnovabili e per continuare a collaborare con una azienda multinazionale mi fu chiesto di creare la filiale italiana per la vendita e l'installazione di impiandi di produzione di energia elettrica dal sole. Avevo una certa esperienza tanto che ero sdivntato il responsabile di un forum sulla produzione di energia elettrica dal sole. RIcordo le mitiche battaglie dialettiche con il professor Battaglia protagonista indiscusso delle lobby nucleariste in italia che adesso stanno tornando in auge. Il fatto di aver fondato quell'azienda fu uno dei più grandi errori di leadership che ho fatto. Il settore era più che promettente (o almeno così sembrava a me (erano i primi anni 2000). Li ho ottenuto un primo vero fallimento che però mi ha insegnato molto. Conoscevo veramente il settore, conoscevo veramente le migliori tecniche di marketing, internet lo bazzicavo dai primi anni 90 anche in modo abbastanza competente eppure fu un buco nell'acqua che preferisco non ricordare. La verità? Quando un settore ti sembra così promettente lo sarà anche per molti altri e in IItalia e non solo purtroppo, il settore dell'energia è soggetto a normative dettate da lobby potenti e da leadership politiche ed econominche che fanno in modo che le cosa non vadano via "liscie", piuttosto la burocrazia sia italiana sia europea tendono a sostenere lo statur quo
\end{verbatim}

\textbf{5. Il fallimento della persuasione} \emph{Hai mai cercato di convincere qualcuno usando solo passione o autorità posizionale, fallendo miseramente? Come è andata e cosa hai capito sulla persuasione?}

\textbf{RISPOSTA:}

\begin{verbatim}
fino ad una certa età, direi i 50 anni, ho sempre immaginato ed ero estremamente convinto del fatto che le persone potessero essere coinvolte usando solo la passione; poi mi sono illuso di poterle coninvolgere con l'autorità, ed inficne ho capito che solo la competenza e il saper coinvolgere le persone al livello di competenza in cui si trovano è la sola cosa plausibile. La persuasione può essere utile ma presuppone che le persone diventino effetto delle tue direttive. in questo modo diventano einefficaci e incompetenti e riescono a crescere solo fino al proprio livello di incompetenza, così come diceva un noto formatore.
\end{verbatim}

\textbf{6. La visione perfetta ignorata} \emph{Racconta di una volta in cui hai creato una ``visione perfetta'' che nessuno ha seguito. Perché secondo te non ha funzionato?}

\textbf{RISPOSTA:}

\begin{verbatim}
La mia visione perfetta, quella che avevo ai 16 anni e che ho cercato in mille modi per metterla in atto era fondata sull0idea che era la vocazione la chiave di volta per trasformare la società. La realtà, la dura realtà è che nessuno ha voglia di seguire altri per le loro vocazione; ogni persona ha una sua vocazione e difficilmente riesce ad esprimerla nei fatti. Bisogna che questa personale vocazione si radichi nella realtà. Non è tanto quello che fai o quello che ottieni a fare la differenza. Piuttosto il fatto che le persone che ti circondino pensino e sentano che tu puoi concedere loro quel potere che loro stessi desiderano e che da soli non riescono ad ottenre. QUesto penso sia la chiave nascosta della leadership
\end{verbatim}

\textbf{7. L'errore del controllo totale} \emph{Hai mai fatto l'errore di cercare di controllare tutto in un team? Qual è stato il risultato disastroso e come hai cambiato approccio?}

\textbf{RISPOSTA:}

\begin{verbatim}
L''inizio della mia impresa di successo ero da solo; pensavo che solo io fossi in grado di far andar bene le cose. Poi mi sono accorto che se "volevo crescere" dovevo trovare il modo di trasmettere questo mio "sapere" anche ad altri. Mi ricordo che da solo e senza nessuna reale utilità scrissi il codice etico della mia azienda Partii dal concetto del global compact e dei 10 principi cardine che lo animano. In fondo erano il minimo comun denominatore a cui tutti potevano dirsi condiscendenti. Sono termini così generali e in qualche misura univarsali che ogni persona potrebbe considerale o le poteva considerare a qule tempo condivisibili. La realtà la vediamo oggi, un tempo in cui questi concetti non vengono applicati che nel cosiddetto "occidente" e completamente disattesi dalla maggioranza dei popoli e delle società. Una gran delusione accorgersi che un terzo delle perone del mondo siano idealisti occidentali e i tre terzi lottino per la sopravvivenza tramite la forza
\end{verbatim}

\begin{center}\rule{0.5\linewidth}{0.5pt}\end{center}

\subsection{🚀 SEZIONE III: I BREAKTHROUGH MOMENTS}\label{sezione-iii-i-breakthrough-moments}

\subsubsection{\texorpdfstring{\textbf{I MOMENTI DI SVOLTA}}{I MOMENTI DI SVOLTA}}\label{i-momenti-di-svolta}

\textbf{8. La prima azione senza autorizzazione} \emph{Descrivi il primo momento in cui hai scelto di agire da leader senza aspettare autorizzazione esterna. Cosa è successo e come ti sei sentito?}

\textbf{RISPOSTA:}

\begin{verbatim}
La scittura della mia prima lettera di intenti la ho fatta grazie a covey nel 1986. Quella scelta dettata dal riconoscere che il libro i 7 pilastri di covey sulla proattività mi aveva cominciato a dare degli strumenti concreti che potevo utilizzare sia in piccolo sia in grande, il quadro di eisenower sulle priorità, e gli altri strumenti che presentava covey sono risultate un punto importante in cui ho capito che non era la voglia di creare un camvbiamento a poter dettare una strada percorribile per diventare un leader di me stesso e di altri. Già allora, già da molto prima avevo scelto di cambiare le cose in modo pragmatico ma covey raccolse in qull'unico libro alcuni deglis tumenti che ancor oggi consideo essenziali alla leadership e che mettono il lettore a punto causa rispetto alla propria vita e a quella degli altri. FOrte di questa mia consapevolezza ho provato a convincere altre persone del mio enturage a far andare le cose in verso giusto. Ma la cosa fu un vero fallimento che confluì nel racconto drammatico della cabina telefonica che ho fatto prima. Ho provato poi a fare di "testa mia" e queso mi è costato molto fino al punto di essere riuscito a creare il mio "colpo di fortuna" cha ha cambiato finalmente il mio approccio alla realtà. Sono dovuto diventare pragmatico; son dovuto diventare IL PROTAGONISTA della mia stria; è sta una scelta inconsapevole, forse azzartata, fa fino a quando non la ho fatta non ho potuto "incidere" concretamete nella realtà. Fino a quando devi chiedere il permesso sei a punto effetto, sei un bersaglio e non sei la vera sorgente del cambiamento che vuoi incarnare
\end{verbatim}

\textbf{9. La scoperta della struttura} \emph{Quando hai capito per la prima volta che la leadership aveva una ``struttura'' replicabile? Cosa ha scatenato questa comprensione?}

\textbf{RISPOSTA:}

\begin{verbatim}
SOno sempre stato affascinato dalle strutture; avendo studiato chimica organica avevo sempre immaginato che strutture stabili potevano essere il contenitore di qualsiasi composto organico ed inorganico. Per questo ho semtre visto nelle strutture di docuenti e nella strutturazione dell'ossatora di una a zienda tramite il codice etico, ad esempio, fosse fondamentale. Lo stesso global compact non è che una struttura di principi generali che possono essere declinati in miglioni  se non milliardi di modi. La struttura di pensiero, le prelogiche e le logiche, i fattori sonoilssottofondo a cui la vita si attacca per esprimenrsi, fiorire e prosperare. Quando lessi il libro di Covey la proattività mi sembò essere il tappeto di base su cui si poteva esprimere la leadeship e d ho riconosciuto la prima struttura utile nel quadro di eisenower. Poi da li è partito il mio studio fino a quando incontrai quasi per caro i 24 principi della genialità che considero la summa delle caratterisciche dei leader a cui tutti in un modo più o meno ampio o parziale (e non è questo il puto) si trovano a costruire la propria leadeship. La verià che la leadership è una abiltà che si può acquisieìre e potenziare. Io stesso ne sono un esempio calzante. Queste caratteristiche delle personalità sono insite in ogni carattere personale. Per diventare delle abiltià devono ottenre un nuovo livello di consapevolezza. Proprio come si impara ad andare in bicicletta la prima volaìta sìdevi praticare l'equilibrio per poter arrivare a pedalare e la stessa cosa nella leadership. Esiste la leadership e le caratteriscitiche inanate di ognuno di noi ne cositituiscono lossatura. Un approccio pragmatico può far acquisire consapevolezza delle proprie peculirità per innalzarle alla leadeship che essenzialment e profondamente fanno parte del costro essere sociali, nel nostro desiderio più o meno espresso di collaborare insieme per raggiungere una capacità infinita di fiorire e prosperare
\end{verbatim}

\textbf{10. Il primo approccio sistematico} \emph{Racconta la prima volta che hai applicato un approccio sistematico (invece di improvvisare) in una situazione di leadership. Quali sono stati i risultati?}

\textbf{RISPOSTA:}

\begin{verbatim}
L aprima volta che ho scritto la mia dichiarazione di intenti era semplicemente per capire profondamente le esigenze e l'allineamente di me e della mia compagna. Devo dire che è stato illuminante capire come due menti interpretano in modo diverso la stessa realtà . QUel momento di "comunione" dei diversi intenti mi ha fatto veramente capire, e forse per la prima volta, che non è possibile coinvolgere allo stesso modo le menti, il cuore, la volontà, la determinazione (e potrei andare aventi con tutte le altre elencazioni delle 24 caratteristiche dei geni) di ogni persona che mi circontdano. La vita è protagonismo personale applicato a mete, scopi, policy che devono essere accettate ed accomunate tra le persone per poter arrivare alla collaborazione. Solo allora potrebbe diventare plausibile un piano di conquista, laììil coinvolgimento in programmi e progetti. Solo allora una persona può accettare di ricevere in modoalità sorgente un ordine e farla sua verso il raggiungimento di una scena ideale condivisa. I prodotti e i servizi, devono essere indirizzati alla soddisfazione di un pubblico secifico, solo in quel mosoìdo possono essere accettati e generare delle KPI , delle statisctiche di espansione.
\end{verbatim}

\textbf{11. Quando tutto si è incastrato} \emph{C'è stato un momento in cui tutti i ``pezzi del puzzle'' della leadership si sono incastrati insieme? Descrivilo nel dettaglio.}

\textbf{RISPOSTA:}

\begin{verbatim}
Quando cominci a vedere che i numeri e le statistiche ti danno ragione, quando un'espansione della consegna ti da ragione: ecco quello è il momento in cui ti accorgi che il tuo modo di intrptretare il tuo ruolo è diventato una leadescìhip risolutiva che anche altri possono acettare e celebrare condividendone la visione. Questo per me è avvenuto nel Marzo del 2007, abbastanza avanti nell'età ma un punto di svolta non è frutto di un unica azione. SOno migliaia le scelte, gli errori, le vittorie che la costruiscono. Io posso dire di essere sempre stato il Leader di me stesso; gi a 16 anni avevo la sensazione di poter fare la differenza in positivo. Ma i tentativi, i fallimenti, le vittorie accumulate giorno dopo giorno, anno dopo hanno hanno trovato la loro apoteosi in un momento di solitudine quasi casuale, quando quasi per scommessa sono entrato nel settore dell'oro e della gioielleria. Ecco nel 2007, esattamente nel mmese di marzo, ho incontrato il mio momento di svolta e più nulla è mai stato come prima. Il successo, quasi in modo eclatante è diventato il mio modus operandi. Ogni cosa che ho fatto da allora non ha fatto altro che riaffermare che diverse caratteristiche di un leader che ho piano piano, vittoria dopo vittoria, rafforza ed ancora sto rafforzando.
\end{verbatim}

\begin{center}\rule{0.5\linewidth}{0.5pt}\end{center}

\subsection{🏗️ SEZIONE IV: LA COSTRUZIONE PRATICA}\label{sezione-iv-la-costruzione-pratica}

\subsubsection{\texorpdfstring{\textbf{I TUOI ESPERIMENTI SUL CAMPO}}{I TUOI ESPERIMENTI SUL CAMPO}}\label{i-tuoi-esperimenti-sul-campo}

\textbf{12. Il primo esperimento di sviluppo talenti} \emph{Racconta del primo team o progetto dove hai applicato consapevolmente i principi di sviluppo del talento negli altri. Come è andata?}

\textbf{RISPOSTA:}

\begin{verbatim}
Il tsalento delle persone non ha bisogno di essere coltivato ma solo riconosciuto. Ogni persona ha qualcosa da dire a riguardo di specifici argomenti e specializzazioni. Ricordo con piacere una persona di una crìerta età che mi ha dimostrato il suo apprezzamento nei fatti e lei stessa ha coltivato la mia fiducia insegnandomi cose che non sapevo. Alla fine il dare e l'avere è alla base della vitalità. La fiducia reciproca, il lasciar fare a chi sa qualcosa più di te e riconoscelo; questo tipo di umiltà che concede di essere, che non impone partecipazione ma la desidera. Sono alla base dello  sviluppo di se e degli altri.
\end{verbatim}

\textbf{13. La trasformazione del team difficile} \emph{Hai mai trasformato un team ``difficile'' o disfunzionale? Quali erano i problemi iniziali e quali passi concreti hai fatto?}

\textbf{RISPOSTA:}

\begin{verbatim}
Ogni volta che ho dovuto affrontare un approccio non funzionale, alla fine di tutto ho capito che la comunicazione sincera fa sempre la differenza. Come compresi la prima volta che feci una dichiarazione di intenti, ognuno percepisce la realtà in modo diverso; ognuno ha dei dettagli della realtà che lo colpiscono in modo diverso. Quello che è importante per uno per un altro è insignificante e viceversa. Quando diventi consapevole di questa grandi differenza nella percezione della realtà capisci anche che è l'accordo, ciò su cui si è in accordo l'unica realtà condivisibile. Questo comune accordo ti porta ad apprezzare la vicinanza di persone che ti sono più simili di altre. Ed è in questo modo che si formano le squadre vincenti. L'unità di intenti è sostanziale non solo fondamentale. Ma questa può comunque essere raggiunta se si accettano le differenze di opinione come un punto di vista da cui una persona affronta e risolve le cose come una esigenza della persona. La condivisione l'accordo allora diventano possibili. Ecco è qui che la leadeship diventa potere, fortuna che ti arride, e che nonostante tutto devi volere per poterla ottenere.
\end{verbatim}

\textbf{14. I tuoi primi 90 giorni sistematici} \emph{Descrivi un periodo di 90 giorni in cui hai applicato sistematicamente tutti i tuoi principi di leadership. Cosa è cambiato concretamente?}

\textbf{RISPOSTA:}

\begin{verbatim}
quando inizia un percorso non hai bisogno di mettere una data termine la cosa più importante è iniziarlo, la seconda più importante è l'insistere nonostante qualsiasi cosa si possa incontrare. Poi alla fine i migliori appaiono quelli che riescono a definire tempi e modi ed impegnarsi per il loro raggiungimento. Io ho un MUST che dice "quello che pensi diventa" e ho sempre avuto modi di esserre stupito di quanto sia vero. Se sei determinato , se decidi, e fai il primo passo che tenda a dimostrare, a manifestare nella realtà questo hai gia fatto il passo fondamentale. Poi il passo successivo è perseverare in modo autodeterminato al rggiungimeno. A volte mi è bastato decidere ed iniziare. Mi sono certamente dato una data un target, ma poi me ne sono dimenticato abbandonandomi al flusso autodeterminato e alla fine quello che spesso succede e che la data che pensavi quasi irraggiungibile si è dimostrata come la data del tuo raggiungimento. Che fosse definitivo non è importante. Quello che è importante è la  decisione, l'impegno, la determinazione al risultato. Poi passo dopo passo conseguentemente, otterrai ul tuo punto di arrivo. Quello sarà anche il nuovo punto di ripartenza. Ecco come si ha ed ottiene una contina aggressione e trasformazione della realtà in quello che vuoi tu.
\end{verbatim}

\begin{center}\rule{0.5\linewidth}{0.5pt}\end{center}

\subsection{🎭 SEZIONE V: I PERSONAGGI DELLA TUA STORIA}\label{sezione-v-i-personaggi-della-tua-storia}

\subsubsection{\texorpdfstring{\textbf{LE PERSONE CHE HANNO SEGNATO IL TUO PERCORSO}}{LE PERSONE CHE HANNO SEGNATO IL TUO PERCORSO}}\label{le-persone-che-hanno-segnato-il-tuo-percorso}

\textbf{15. L'anti-mentore} \emph{Chi è stato il tuo ``anti-mentore'' - la persona che ti ha mostrato come NON essere un leader? Cosa faceva di sbagliato?}

\textbf{RISPOSTA:}

\begin{verbatim}
SIcurmamente un anti leader è stata una persona che pensava solo al proprio tornaconto e che in modo iocrita mostrava il contrario
\end{verbatim}

\textbf{16. Chi ti ha superato} \emph{Racconta di qualcuno che hai sviluppato come leader e che poi ti ha superato. Come hai vissuto quel momento?}

\textbf{RISPOSTA:}

\begin{verbatim}
SIcuramente una persona giovane, fresca di università mi ha dimostrato che alcune persone hanno solo bisogno di supporto in momenti topici, ma poi alla prova dei fatti vedi che quelle persone speciali esistono e se riesci ad aiutarle ad esprimere il meglio di loro stesse la soddisfazione è 100 volte più grande di una vittoria economica
\end{verbatim}

\textbf{17. Lo specchio umano} \emph{C'è stata una persona che ti ha fatto da specchio, mostrandoti un aspetto della leadership che non vedevi? Cosa ti ha fatto capire?}

\textbf{RISPOSTA:}

\begin{verbatim}
SI spesso ho potuto incontrare persone che avevano sviluppatoo caratteristiche di genialità che in me erano ancora poco sviluppate. Questa mia asservazione, allìinizio avvolta quasi da stupore e meraviglia, mi ha dato modo di migliorare tantissimo in quegli aspetti che ancora oggi trovo difficili da interpretare come personaggio. La tolleranza degli errori, che sono sempre possibili, il desiderio di aiutare comunque e in ogni modo, sono tutte cose che ho osservato prima negli altri e dopo ho cercato di fare mie copiandole e specchiandomi in loro
\end{verbatim}

\begin{center}\rule{0.5\linewidth}{0.5pt}\end{center}

\subsection{🔍 SEZIONE VI: I PATTERN E I PRINCIPI}\label{sezione-vi-i-pattern-e-i-principi}

\subsubsection{\texorpdfstring{\textbf{LE TUE SCOPERTE ORIGINALI}}{LE TUE SCOPERTE ORIGINALI}}\label{le-tue-scoperte-originali}

\textbf{18 La genesi dei Quattro Pilastri dell'Autorevolezza} \emph{Come hai scoperto o sviluppato il concetto dei ``Quattro Pilastri dell'Autorevolezza'' (Visione, Azione, Relazioni, Adattamento)? C'è stata una situazione specifica che ti ha illuminato sulla necessità di questa struttura a quattro pilastri?} ?*

\textbf{RISPOSTA:}

\begin{verbatim}
mi ricordo che anni fa decisi di dare un motto alla mia firma. Il motto era "Soluzioni, Azioni, Risultati"; questo fu il primo embrione che nel tempo e con ulteriori studi si è cristallizzato nei 4 pilastri che compongono questo libro. La dichiarazione di intenti di covey era stato il primo passo per riconoscere la necessità di una visione personale e collettiv per costruire un gruppo che collabora e ottenere dei risultati. Nulla della visione crea una trasformazione della realtà in positivo, solo l'Zazione trasforma l'adesso in possibilità e nella creazione di un futuro migliore. Le relazioni poi, poco alla volta sono diventate una essenzialità fondamentale. Per assurdo anche una relazione non proprio sana, anche con alcune caratteristiche tossiche, in alcuni momenti è meglio di essuna relazione. I gruppi creano un maggior potere del singolo, è come se in qualche misura potessereavere una vita propria. Come del resto l'umanità intera esiste come entità nonostante i singoli. L'adattare la realtà poi è il pilastro fondamentale della trasformazione in positivo del mondo. Cambiare le cose è un accadimento che deve prima essere visto, poi deve essere attuato e insieme a questo sviluppare delle relazioni che ampliano l'espansione. La trasformazione adattiva della realtà deve comunque partire dal contingente. Come dice un amico bisogna che esiste una condizione che ti chiama alla trasformazione per poi riuscire ad adattare ad una nuova realtà quella vecchia che consideravi una condizione poco accettabile.
\end{verbatim}

\textbf{19. Dal servizio all'ego} \emph{Quando hai capito che la leadership autentica era ``servizio'' e non ``ego''? Racconta l'episodio che ti ha umiliato e poi trasformato.}

\textbf{RISPOSTA:}

\begin{verbatim}
Passare da autoritarismo ad autorità nei fatti è un passaggio fondamentale da fare e per me è accaduto nel momento in cui le cose non andavano bene. Mi sentivo quasi in pericolo (come le cose non vanno coe ho deciso?) questa era la domanda che mi hja costretto ad affrontare i miei limiti e a migliorare nonostante io pensassi di essere già perfetto. La verità e la realtà, se sei attento a quello che ti accade e lo accetti come insegnamento positivo, ti da tutti gli strumenti per diventare la migliore versione di te stesso di continuo. Il continuo e costante miglioramento, non è un assunto manageriale alla Porter è una verità che si manifesta nella natura delle cose e dell'universo. Alla faccia della legge dell'entropia che nel mondo dello spirito umano non ha ragione di essere considerata.
\end{verbatim}

\textbf{20. Smettere di clonare se stessi} \emph{Hai mai avuto un momento in cui hai realizzato che stavi ``clonando te stesso'' invece di sviluppare l'autenticità degli altri? Come hai corretto il tiro?}

\textbf{RISPOSTA:}

\begin{verbatim}
QUesto è l'ostacolo principale alla vera leadership. Fino ad un certo punto nella mia vita ho cercato di "clonare" me stesso (essere perfetto, scusate la battuta ma è quello che a volte capita di pensare soprattuto da adolescenti) neegli altri. Ma solo quando ho scoperto che ogni persona ha un proprio punto di vista da cui osserva la stessa mia realtà, che ho cominciato ad accogliere l'altro e a crescere come leader di un gruppo di collaboratori di successo.
\end{verbatim}

\begin{center}\rule{0.5\linewidth}{0.5pt}\end{center}

\subsection{🌍 SEZIONE VII: IL CONTESTO E L'AMBIENTE}\label{sezione-vii-il-contesto-e-lambiente}

\subsubsection{\texorpdfstring{\textbf{IL TUO CAMPO DI BATTAGLIA}}{IL TUO CAMPO DI BATTAGLIA}}\label{il-tuo-campo-di-battaglia}

\textbf{21. Il tuo settore di riferimento} \emph{In quale settore/industria hai principalmente applicato questi principi? Quali sono le sfide specifiche di leadership in quel contesto?}

\textbf{RISPOSTA:}

\begin{verbatim}
Io ho una micro impresa e quindi il mio ambito di applicazione non deve sottostare a forze e lotte di potere enormi, allo status quo dei grandi gruppi di persone e alle lotte dipoere per accrescere all'interno di un gruppo troppo grande. Per mia esperienza i gruppi non devono essere troppo grandi. Il proprio entourage diretto deve essere più simile ad una scquadra di calcetto che essere simile ad un'orda di calcio fiorentino. I gruppi nel complesso possono essere anche molto cìgrandi ma il proprio enturage immediato non dovrebbe essere più grande delle 7 o 8 persone.
\end{verbatim}

\textbf{22. Differenze culturali} \emph{Hai mai applicato questi principi in un contesto culturale diverso da quello italiano? Cosa hai scoperto sulle differenze culturali nella leadership?}

\textbf{RISPOSTA:}

\begin{verbatim}
Sicuramente un gruppo di tedeschi, e un gruppo di americani sono completamente diversi da un gruppo di italiani. GLi americani sono pragmatici e non tendono ad approfondire come gli italiani le relazioni tra le persone. I tedeschi hanno un modo di collaborare molto più gerarchico di quello italiano dove la personalità dei singoli è fondamentale.
\end{verbatim}

\textbf{23. Il progetto più ambizioso} \emph{Qual è stato il progetto o l'obiettivo più ambizioso che hai mai guidato? Come hai gestito la pressione e l'incertezza?}

\textbf{RISPOSTA:}

\begin{verbatim}
Io ho sicuramente progetti ambiziosi, sin da piccolo li ho avuto. Per ora quello più ambizioso è sempre stato l'ultimo su cui ho lavorato. Lla mia ambizoine personale è sempre alta manon sempre quello che determino è grande per un'altra persona. Per me è sempre grande quello che faccio
\end{verbatim}

\begin{center}\rule{0.5\linewidth}{0.5pt}\end{center}

\subsection{🎯 SEZIONE VIII: IL FUTURO CHE STAI COSTRUENDO}\label{sezione-viii-il-futuro-che-stai-costruendo}

\subsubsection{\texorpdfstring{\textbf{LA TUA VISIONE IN AZIONE}}{LA TUA VISIONE IN AZIONE}}\label{la-tua-visione-in-azione}

\textbf{24. La tua evoluzione attuale} \emph{Su cosa stai lavorando adesso che rappresenta la tua evoluzione più recente come leader? Quali nuove sfide stai affrontando?}

\textbf{RISPOSTA:}

\begin{verbatim}
La delega spinta, il riuscire a gestire due azienda da lontano, molto lontano mi sta dando successi inimmaginabile. Pare che più io mi allontani dalle mie azienda (senza ma perderne il controlla naturalmente) più il successo mi arride. In ultimo VITAEOLOGY è sicuramente il progetto che corona la mia carriera e che è oggi il più sfidante che abbiammai avuto
\end{verbatim}

\textbf{25. Il legacy che costruisci} \emph{Cosa vuoi che rimanga del tuo impatto come leader? Qual è il ``legacy'' che stai costruendo?}

\textbf{RISPOSTA:}

\begin{verbatim}
La capacità di fare la differenza in meglio in qualsiasi ambito io mi avvicini e colga e desideri di creare un impatto positivo e migliorativo
\end{verbatim}

\textbf{26. Consiglio al te del passato} \emph{Se dovessi dare un consiglio al ``te'' di 10 anni fa che stava iniziando questo percorso, cosa gli diresti?}

\textbf{RISPOSTA:}

\begin{verbatim}
Anche se oggi non hai esattamente chiaro dove arriverai la cosa più importante è iniziare e perseverare nonostante tutto
\end{verbatim}

\begin{center}\rule{0.5\linewidth}{0.5pt}\end{center}

\subsection{🔥 SEZIONE IX: I DETTAGLI CHE CREANO CONNESSIONE}\label{sezione-ix-i-dettagli-che-creano-connessione}

\subsubsection{\texorpdfstring{\textbf{I PARTICOLARI CHE RENDONO REALE LA STORIA}}{I PARTICOLARI CHE RENDONO REALE LA STORIA}}\label{i-particolari-che-rendono-reale-la-storia}

\textbf{27. La routine che ti ha cambiato} \emph{Qual è stata la tua routine o abitudine che ti ha aiutato di più nel sviluppare la leadership? Come l'hai scoperta?}

\textbf{RISPOSTA:}

\begin{verbatim}
Acoltare fino in fondo le istanze dei collaboratori e delegare loro la responsabilità di scelte e risultati
\end{verbatim}

\textbf{28. L'oggetto/luogo simbolico} \emph{C'è un oggetto, un posto o un rituale che associ al tuo sviluppo come leader? Racconta il significato.}

\textbf{RISPOSTA:}

\begin{verbatim}
fare una passeggiata in solitudine in qualsiasi posto sia possibile fare una passeggiata e staccare dlle routine
\end{verbatim}

\textbf{29. La critica più dura} \emph{Qual è stata la critica più dura che hai ricevuto sul tuo stile di leadership? Chi te l'ha fatta e come hai reagito?}

\textbf{RISPOSTA:}

\begin{verbatim}
l'autoritarismo che avevo e che ho dovuto per cforza di cose mitigare per diventare una autolità nel mio gruppo di collaboratori
\end{verbatim}

\textbf{30. Il momento di solitudine} \emph{Racconta un momento di solitudine o dubbio nel tuo percorso di leadership. Come hai superato quel periodo?}

\textbf{RISPOSTA:}

\begin{verbatim}
Ricordo un momento fondamentale nella mia adolescenza che ho dovuto accettare compromessi che non mi sarei mai aspettato di accettare
\end{verbatim}

\begin{center}\rule{0.5\linewidth}{0.5pt}\end{center}

\subsection{📚 SEZIONE X: I COLLEGAMENTI STORICI}\label{sezione-x-i-collegamenti-storici}

\subsubsection{\texorpdfstring{\textbf{I TUOI MODELLI E ANTI-MODELLI}}{I TUOI MODELLI E ANTI-MODELLI}}\label{i-tuoi-modelli-e-anti-modelli}

\textbf{31. Il leader storico affine} \emph{Quale leader storico senti più affine al tuo approccio e perché? C'è un episodio della sua vita che rispecchia una tua esperienza?}

\textbf{RISPOSTA:}

\begin{verbatim}
Simon Bolivar , anche se ha vissutoin un momento storico in cui le realtà erano molto diverse dal mio; poi igli eploratori conquistatori, e persone comeAntony Robbins e SInnek
\end{verbatim}

\textbf{32. L'anti-modello storico} \emph{C'è un leader storico che ti ha insegnato cosa NON fare? Quale errore suo hai evitato grazie al suo esempio negativo?}

\textbf{RISPOSTA:}

\begin{verbatim}
Smon Bolivar, fidarsi delle persone senza osservarne i risultati
\end{verbatim}

\textbf{33. La tua trinità di leadership} \emph{Se dovessi scegliere 3 leader storici che rappresentano i 3 aspetti più importanti della tua filosofia di leadership, chi sceglieresti e perché?}

\textbf{RISPOSTA:}

\begin{verbatim}
Lao Tze, Edison, GIorgia Meloni
\end{verbatim}

\begin{center}\rule{0.5\linewidth}{0.5pt}\end{center}

\subsection{📝 NOTE FINALI}\label{note-finali}

\subsubsection{\texorpdfstring{\textbf{ELEMENTI AGGIUNTIVI}}{ELEMENTI AGGIUNTIVI}}\label{elementi-aggiuntivi}

\textbf{Momenti che vuoi assolutamente includere:}

\begin{verbatim}
[Aggiungi qui qualsiasi altro episodio, insight o momento significativo che vuoi sia parte della storia...]
\end{verbatim}

\textbf{Personaggi chiave da sviluppare:}

\begin{verbatim}
[Lista delle persone importanti nella tua storia che potrebbero diventare personaggi ricorrenti...]
\end{verbatim}

\textbf{Dettagli emotivi specifici:}

\begin{verbatim}
[Sensazioni, luoghi, oggetti, routine che rendono la storia più vivida e relatable...]
\end{verbatim}

\begin{center}\rule{0.5\linewidth}{0.5pt}\end{center}

\emph{🎯 \textbf{OBIETTIVO}: Una volta completato, questo documento diventerà la base per costruire l'arco narrativo autobiografico che attraverserà tutto il libro, creando quella connessione autentica che trasforma un manuale teorico in un'esperienza di trasformazione personale condivisa.}

\backmatter
\end{document}
